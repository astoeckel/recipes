\documentclass[11pt,a4paper]{article}
\usepackage[utf8x]{inputenc}
\usepackage{ucs}
\usepackage{amsmath}
\usepackage{amsfonts}
\usepackage{amssymb}
\usepackage{graphicx}
\usepackage{csquotes}
\usepackage{xfrac}
\usepackage[ngerman]{babel}
\usepackage{multicol}
\usepackage{microtype}

\usepackage[light,math]{anttor}
\usepackage[T1]{fontenc}


\setlength{\parindent}{0em}
\setlength{\parskip}{0em}
\renewcommand{\baselinestretch}{1.25}

\usepackage[left=3cm, right=3cm, top=3cm, bottom=4cm]{geometry}

\author{Andreas Stöckel}
\begin{document}
	\pagestyle{empty}
	{\Large\bf\centering
	\rule{5cm}{1pt}\\
	Andreas' Rekonstruierter Apfel-Streusel-Auflauf\\\enquote{Apple Crumble}\\[-0.25cm]
	\rule{5cm}{1pt}\\
	}

	\begin{multicols}{2}
	\textbf{Zutaten:}\\
	\textit{Für die Füllung:}
	\begin{itemize}
		\item 3 Esslöffel Speisestärke
		\item 1 Zitrone
		\item $\sfrac{1}2$ Teelöffel Zimt
		\item 900g Äpfel\\(Granny Smith, acht kleinere Äpfel)
		\item 1 Handvoll Nüsse\\(z.B. Walnüsse, Pekannüsse)
		\item 1 Handvoll Rosinen
		\item 3 Esslöffel Demerara-Zucker
	\end{itemize}
	\textit{Für den Belag:}
	\begin{itemize}
		\item 125g Margarine (gesalzen)
		\item 125g Demerara-Zucker
		\item $\sfrac{1}2$ Teelöffel Zimt
		\item $\sfrac{1}2$ Teelöffel Backpulver
		\item 150g Vollkornmehl
		\item 75g Haferflocken\\(schnellkochend, z.B. Köllnflocken)
	\end{itemize}
	\vfill
	\columnbreak
	\textbf{Zubereitung:}
	\begin{enumerate}
		\item Saft einer \textbf{Zitrone} mit \textbf{Speisestärke} und \textbf{Zimt} mischen und glattrühren.
		\item \textbf{Äpfel} schälen, entkernen und in kleine Stücke schneiden. Apfelstücke sogleich in eine Schüssel mit der Zitronensaftmischung geben und gelegentlich umrühren.
		\item \textbf{Nüsse} zerhacken und zusammen mit den \textbf{Rosinen} und dem \textbf{Zucker} unter die Äpfel mischen.
		\item \textbf{Margarine}, \textbf{Zucker}, \textbf{Zimt} und \textbf{Backpulver} in einer Rührschüssel mit einem Rührgerät durchmischen.
		\item \textbf{Mehl} und \textbf{Haferflocken} hinzugeben, mit dem Rührgerät kurz mischen bis große Streusel entstehen.
		\item Füllung in eine gefettete Auflaufform geben, mit Streuseln bedecken. Auflauf mit Aluminiumfolie abdecken.
		\item Auflauf bei 175°C Unterhitze auf mittlerer Schiene ca. 30~Minuten lang backen, danach im Backoffen noch ca. 10 Minuten weiter köcheln lassen.
		\item Schmeckt besonders gut heiß mit Vanilleeis.
	\end{enumerate}
	\end{multicols}
\end{document}